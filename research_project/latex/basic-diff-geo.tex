

\section{Background Theory}%
\label{sec:differential_geometry}

We may define $\Gamma \left( t \right) $ to be a time evolutionary, smooth compact and oriented surface with no boundary in $\mathbb{R} ^{3}$. We will denote the normal unit vector outer normal vector of $\Gamma \left( t \right) $  to be  $
\boldsymbol{\nu} \left( \mathbf{x} \right) $ for $\mathbf{x} \in \Gamma  $ .  Now, let $v$ be some vector field defined in
$\mathcal{V} \in  \mathbb{R}^{3}  $ s.t. the vector $\mathbf{v} = v \boldsymbol{\nu } $  is describing the normal component deformation velocity of the surfaces $\left\{ \Gamma \left( t \right)  \right\}_{t=0}^{T} $.

Let us now define a arbitrary energy functional containing some the surface dynamics $\mathbf{v}$ and a evolutionary surface $\Gamma \left( t \right) $. We may apply the $L_{2}\left( \Gamma  \right) $  gradient flow.

To minimize our energy functional of the surface dynamics will we utilize a method called gradient flows. Gradient flows in surface partial differential equation (PDE) are used to solve physical problems where the surface is changing due to some
external force. The PDE describes how the surface changes over time in response to this force, thus allowing us to model real-world phenomena such as fluid and heat flow. The gradient of the surface PDE determines the direction and magnitude of the
change over time, while the PDE itself may contain additional terms that modify or influence the solution. \cite{dogan2007discrete} An alternative approach would be to solve the problem using standard shape optimization techniques using $\Gamma$ as a variable surface \cite{dalphin2014study}.

To be able do develop the evolutionary PDE's we may introduce the gradient flow of the energy functional \eqref{eq:WE}.  Using the definition from \cite{bouchitte2014shape, bonito2010parametric} can we define the shape derivative of some energy
functional $$\mathcal{J} = \int_{\Gamma }^{} \varphi    $$  to be the limit
\begin{equation*}
d\mathcal{J} \left( \Gamma; w  \right)  = \lim_{t \to 0} \frac{\mathcal{J}\left( \Gamma \left( t \right)  \right) - \mathcal{J} (  \Gamma \left( 0 \right))}{t} \quad \forall w \in  \mathcal{V}
.\end{equation*}
Where it is for our case $\varphi = H ^2$.







