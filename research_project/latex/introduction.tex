\section{Introduction}\label{sec:introduction}


% Why are cell membrane dynamics important

The value of understanding the basic underlying mechanics of the cell membrane dynamics has quite a lot of application. First can we detect deceases such as Alzheimer's disease, cancer cells and develop new methods and vaccines \cite{small2006sorting}.

One of the primary components of the cell membranes are lipids which serve many different functions. A key function is that it is consisting of a bilayer of lipids which controls the structural rigidity and the fluidity of the membrane \cite{ neidleman87}. It also turns out that the lipids often accumulate into so-called lipid rafts which serves as a rigid platform for proteins with special properties such as intracellular trafficking of lipids and lipid-anchored proteins \cite{Edidin03}.

Modelling of lipid rafts formation can be modelled as a two-phase separation problem based on minimization of the Ginzburg-Landau energy functional \cite{yushutin19}
\[
\mathcal{E}_{ch}  \left( \Gamma  \right) = \int_{\Gamma  }^{}\Psi \left( c \right) + \frac{\gamma}{2} \left\lvert \nabla c \right\rvert^{2} dx,
\]
which is describing the chemical energy for a concentration $c: \Gamma \times \left[ 0,T \right] \mapsto  \left[ 0,1 \right]  $ over a surface membrane $\Gamma$. Several authors have solved this problem often results by deriving variants of Cahn
Hilliard Equation or Allen Cahn Equation if the concentration is not conserved both standstill and evolving domains \cite{yushutin19, ratz16,Gera2017, caetano21,yushutin19} .

Assuming that the system is a single-phase system can the elastic bending energy be modelled using the Canham Helrich energy functional \cite{wang08} \[
\mathcal{E} _{e}\left( \Gamma  \right) =   \int_{\Gamma }^{}  2 c_{b} H^{2} + c_{k} K  dx
\].

Here is $H$ denoted as the mean curvature and $K$ as the gaussian curvature with respectively $c_{b}$ and $c_{k}$ as tuning parameters.


















