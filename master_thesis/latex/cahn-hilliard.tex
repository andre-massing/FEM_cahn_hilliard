
\newpage
\section{Cahn Hilliard Equation on a Closed Membrane}%
\label{sec:cahn_hilliard_equation}


Let $c_0$ and $c_1$  indicate the concentration profile of the substances in a $2$ -phase system such
that $c_0 \left( \mathbf{x},t \right): \Omega  \times \left[ 0, \infty \right] \to \left[ 0,1 \right]$ and
similarly $c_1 \left( \mathbf{x},t \right): \Omega \times \left[ 0, \infty \right] \to \left[ 0,1 \right]$, where
$\mathbf{x} $ is a element of some surface $\Omega $ and $t$ is time.
However, in the $2$ phase problem will we will restrict ourself so that $c_0\left( t,\mathbf{x} \right) + c_1\left( t,
\mathbf{x} \right) = 1$ at any $\mathbf{x} $ at time $t$. A property of the restriction is that we now can express
$c_0$ using $c_1$, with no loss of information. Hence, let us now define $c = c_0$ so $c \left( \mathbf{x},t \right):
\Omega  \times \left[ 0, \infty \right] \to \left[ 0,1 \right]$. It has been shown that $2$ phase system if
thermodynamically unstable can be evolve
into a phase separation
described by a evolutional differential equation \cite{cahnhilliard1957} using a model based on chemical energy of the
substances. However, further development has been done \cite{yushutin19} to solve this equation on surfaces. Now assume
model that we want to describe is a phase-separation on a closed membrane surface $\Gamma $, so that $c \left( \mathbf{x},t \right):
\Gamma \times \left[ 0, T \right] \to \left[ 0,1 \right]$. Then is the surface Cahn Hilliard equation described such that

\begin{equation}
    \label{eq:cahn1}
\rho \frac{\partial c}{\partial  t}  - \nabla_{\Gamma } \left( M \nabla _{\Gamma } \left( f_{0}'  - \varepsilon ^2
        \nabla^2
_{\Gamma } c \right) \right) = 0  \quad \text{on } \Gamma
.\end{equation}

We define here the tangential gradient operator to be $\nabla _{\Gamma } c = \nabla c - \left( \mathbf{n} \nabla c
\right)\mathbf{n} $ applied on the surface $\Gamma $ restricted to $\mathbf{n} \cdot \nabla _{\Gamma } c = 0$.

Lets define $\varepsilon $ to be the size of the layer between the substances $c_{1}$ and $c_{2}$. The density $\rho $ is
simply defined such that $\rho = \frac{m}{S_{\Gamma }}$ is a constant based on the total mass divided by the total
surface area of $\Gamma $.
Here is the mobility $M$ often derived such that is is dependent on $c$ and is crucial for the result during a possible
coarsening event \cite{yushutin19}.  However, the free energy per unit surface
$f_{0} = f_{0}\left( c \right)$ is derived based on the thermodynamical model and should according to \cite{yushutin19} be non convex and
nonlinear.

A important observation is that equation \eqref{eq:cahn1} is a fourth order equation which makes it more challenging to
solve using conventional FEM methods. This clear when writing the equation on the equivalent weak form and second order
equations arise.


\newpage
\section{Energy Functionals}%
\label{sec:energy_functionals}
Let $c\left( x,t \right)  : \Gamma \times [0,T] \mapsto [0,1] $. From \cite{yushutin19} can we observe the energy functionals
\begin{equation*}
   E _{1}(c) = \int_\Gamma f(c)
.\end{equation*}
where \[
f\left( c \right) = f_{0}\left( c \right) + \frac{1}{2}\varepsilon ^{2} \left\lvert \nabla _{\Gamma } c \right\rvert ^{2}
\]
and the conservation law  $\rho \frac{\partial c}{\partial  t}  + div_{\Gamma } \mathbf{j} = 0$ for the evolution of  $c $, derived from the Ficks Law $\mathbf{j} = - M \nabla _{\Gamma } \mu $ for the chemical potential derived by the functional derivative $\mu = \frac{\delta
f}{\delta c} $. The double well function is denoted as \[
f_{0}\left( c \right)  = \frac{\zeta}{4} c^2(1- c)^2
\]





\newpage

