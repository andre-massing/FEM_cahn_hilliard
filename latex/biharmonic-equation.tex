\newpage
\section{ $C^0$ Interior Penalty Method for Biharmonic Equation}
\label{sec:ch1}

We will now try to formulate and analyze a similar discontinious Galerkin approach for the biharmonic equation


\subsection{Hybrid DG Biharmonic Equation}%
\label{sub:hybrid_dg_biharmonic_equation}

Let us again define the problem

\begin{equation}
\label{eq:HDG_bi_problem}
\begin{split}
    \nabla ^4 u & = f \quad \text{in } \Omega   \\
    \partial _{n} u & = \partial _{n} \nabla ^2 u = 0 \quad \text{on } \partial \Omega   \\
\end{split}
.\end{equation}

In fact, we must also assume the solvability condtion $ \int_{\Omega }^{} f dx = 0$ to obtain a unique solution according to
equation (2.4) in Brenner \cite{brenner2012}, which also can be related to equation (3.4) in Gu \cite{gu2012c0}.

Anyhow,
let us first define the Hilbert space of the discrete solution \[
\begin{split}
    H^{1}\left( \mathcal{T}_{h}  \right)  & = \left\{ v \in L_{2} \left( \Omega  \right) : v  \mid _{T} \in H^{1} \left( T
    \right) \forall T \in \mathcal{T}_{h}    \right\} \\
    V &=  \left\{ v \in H^2\left( \Omega  \right) : \partial _{n} v = 0 \text{ and } \partial _{n} \nabla ^2 v = 0 \text{ on } \partial \Omega  \right\} \\
\end{split}
\]
\todo[inline]{ Probably need to work more on this argumentation }

Let us now define $u,v \in V$.

\subsubsection{HDG Method}%
\label{ssub:dg_method}
Let us now define our workspace using the spaces
 \[
\begin{split}
    V &=  \left\{ \left( u, u_{F}  \right): u \in H^{4}\left( \mathcal{T} _{h} \right) \cap H^{1}\left( \Omega  \right)   \right\} \\
    V_{h} &=  \left\{ \left( u, u_{F}  \right) : u \in \mathcal{P} ^{k}\left( T \right) \forall T \in  \mathcal{T} ,
    u_{F} \in \mathcal{P} ^{k}\left( E \right) \forall E \in  \mathcal{F}_{h}    \right\} \\
\end{split}
\]

and the ones including the null drichlet conditions \[
\begin{split}
    V_{0} &= \left\{ \left( u, u_{F} \right) \in  V : \quad u =0, u_{F} = 0 \text{ on } \partial \Omega   \right\}, \\
    V_{0,h} &= \left\{ \left( u, u_{F} \right) \in  V_{h} : \quad u =0, u_{F} = 0 \text{ on } \partial \Omega   \right\}
   . \\
\end{split}
\]
Let $\left( u, u_{F} \right) \in  V $\[
    \begin{split}
\sum_{T}^{} \left( \nabla ^{4} u,v \right) & = \sum_{T}^{} \left( \nabla ^{3} u, \nabla  v \right)_{T} + \left<\partial _{n}
\nabla ^2 u, v \right>_{\partial \Omega } \\
&= \sum_{T}^{} \left( \nabla ^2 u, \nabla ^2 v \right)_{T} - \left<\partial _{n} \nabla u, \nabla v \right>_{\partial T}
+ \left< \partial _{n} \nabla ^2 u, v \right> _{\partial T}\\
    \end{split}
\]


\subsubsection{Basic DG method}%
\label{ssub:basic_dg_method}
Let $w,v \in  H^{4} \left( T  \right) $. Using the same method as in equation (3.6) in  \cite{gu2012c0} can we
deduce that for every triangle $T \in  \mathcal{F}_{h} $ \[
    \begin{split}
        \left( \nabla ^{4} w, v \right) _{T} &= \left< \partial _{n} \nabla ^2 w, v \right>_{\partial T} - \left( \nabla \left( \nabla ^2 w
 \right), \nabla  v  \right)_{T}   \\
 &= \left( D^2w, D^2v \right)_{T} + \left< \partial _{n} \nabla ^2 w, v \right>_{\partial T}  - \left<\partial _{n}
 \nabla w, \nabla v \right>_{\partial T} \\
 &=  \left( D^2 w, D^2 v \right)_{T} - \left<\partial _{nt} w, \partial _{t} v \right>_{\partial T} - \left<\partial
 _{nn} w, \partial _{n} v \right> _{\partial T} +  \left<\partial _{n} \nabla ^2 w, v \right> \\
    \end{split}
\]
Keep in mind that this is a results by defining $\nabla  = \left( \partial _{n}, \partial _{t} \right) $ such that
$\left<\partial _{n} \nabla w, \nabla v \right>_{\partial T} = \left<\partial _{nt} w, \partial _{t} v\right> _{\partial
T} + \left< \partial _{nn} w, \partial _{n} v  \right> _{\partial T} $. Thus, letting $u,v \in
H^{4}\left( T  \right) $  does this hold for local continuity

\begin{equation}
\label{eq:bi_basic_dg}
\left( \nabla ^{4} u,v \right) _{T} = \left( D^2u,D^2v \right) _{T } - \left<\partial _{nt} u, \partial _{t}v
\right>_{\partial t} - \left<\partial _{nn} u, \partial _{n}v \right>_{\partial T} + \left<\partial _{n} \nabla ^2 u,v
\right>_{\partial T}
.\end{equation}

For global continuity does it end up with  so that $v \in \left\{ v \in H^{1}\left( \Omega  \right): v_{T} \in  H^{4}\left( T \right), \ \forall T \in
\mathcal{T}_{h}    \right\}   \cap C^{0} (
\overline{\Omega }  ) $ such that

\begin{equation}
\label{eq:bi_basic_dg_full_1}
\left( \nabla ^{4} u, v \right) _{\Omega }
= \sum_{T \in  \mathcal{T} _{h}}^{} \left( D^2u, D^2v \right)_{T}  + \sum_{E \in
\mathcal{F} ^{ext}_{}}^{} \left<\partial _{n} \nabla  ^2 u, v  \right> _{E}
- \left<\partial _{nt} u, \partial _{n} v \right> _{E}
+ \sum_{E \in \mathcal{F}  ^{int}}^{} \left<\partial _{nn} u , \jump{ \partial _{n_{e}} v }
\right>_{E}
.\end{equation}
(This comes from a similar equation (3.7) given in Gu \cite{gu2012c0}.
What we see is that for \eqref{eq:bi_basic_dg} and \eqref{eq:bi_basic_dg_full_1} to be equivalent on normal and global form must this be true
\[
 \sum_{T \in  \mathcal{T}_{h} }^{}  - \left<\partial _{nt} u, \partial _{t}v
\right>_{\partial T} - \left<\partial _{nn} u, \partial _{n}v \right>_{\partial T} + \left<\partial _{n} \nabla ^2 u,v
\right>_{\partial T}
=  \sum_{E \in
\mathcal{F}^{ext}  }^{} \left<\partial _{n} \nabla  ^2 u, v  \right> _{E}
- \left<\partial _{nt} u, \partial _{n} v \right> _{E}
+ \sum_{E \in \mathcal{F}^{int} }^{} \left<\partial _{nn} u , \jump{ \partial _{n_{e}} v }
\right>_{E}
\]




\begin{enumerate}[label=(\alph*)]
    \item Here is what is happening in Gu \cite{gu2012c0}. Let $w_{h}, v_{h} \in V_{h} = \left\{ v \in C\left(
        \overline{\Omega }  \right): v_{T } = v \mid _{T} \in  \mathcal{P}   _{2} \left( T \right) \quad  \forall T \in
    \mathcal{T}_{h} \right\} $.
Anyhow, \textbf{assuming} that this equation holds can we introduce the numerical correction term,  \[
\sum_{E \in  \mathcal{F} _{h}}^{}  \tau _{h} \left< \jump{ \partial _{n_{e}} w_{h}}, \jump{     \partial _{n_{e}} v_{h}
}   \right>_{E}
\]


\todo[inline]{ Do some research on the correct stability and symmetry term and why this is necessarry.}
Where $\tau _{h}$ is to be determined based on each triangulation.

Keep in mind that the jump is defined as $\jump{ \partial _{ne} v _{h} } = n_{e} \left( \nabla v_{+} - \nabla v_{-}
\right)   $. We have now the basic DG method

\begin{equation}
\label{eq:basic_dg_A_h}
\begin{split}
\mathcal{A} \left( w_{h}, v_{h} \right)   =&  \sum_{T \in \mathcal{T} _{h}}^{} \left( D^2 w_{h}, D^2v_{h} \right) _{T} \\ +
 & \sum_{E \in \mathcal{F}_{h} }^{} \left<\mean{  \partial _{n_{e} n_{e}} w_{h} }, \jump{ \partial _{n_{e}} v_{h}}
\right>_{E}  + \left< \mean{ \partial _{n_{e} n_{e}} v_{h} }, \jump{ \partial _{n_{e}}w }      \right>_{E}
+
\tau _{h} \left< \jump{ \partial _{n_{e}} w_{h}}, \jump{     \partial _{n_{e}} v_{h}   }   \right>_{E}
\end{split}
.\end{equation}
and
\begin{equation}
\label{eq:basic_dg_F_h}
F\left( v_{h} \right)  = \left( f, v_{h} \right) _{\Omega }
.\end{equation}
Hence, the discretized numerical problem is to solve \[
\mathcal{A}\left( w_{h}, v_{h} \right)   = F\left( v_{h} \right), \quad w_{h},v_{h} \in V_{h}  = \left\{ v \in C\left(
        \overline{\Omega }  \right): v_{T } = v \mid _{T} \in  \mathcal{P}   _{2} \left( T \right) \quad  \forall T \in
    \mathcal{T}_{h} \right\} .
\]


In Gu \cite{gu2012c0} ( eq 3.10 p.31) they introduce \[
            \begin{split}
    \left( D^2v: D^2w \right)_{T } & = \left<\partial _{nt} v , \partial _{t} w  \right>_{\partial T} + \left< \partial
    _{nn} v
    \partial _{n}, w \right>_{\partial T}. \\
    &= \sum_{i=1}^{3} \partial _{n_{i} t_{i}} v \int_{E_{i}}^{} \partial _{t} w ds +\left< \partial
    _{nn} v,
    \partial _{n} w \right>_{\partial T}     \\
    &=\left< \partial
    _{nn} v,
    \partial _{n} w \right>_{\partial T}  \\
            \end{split}
    \]
    Using the identity $ab +cd = \frac{1}{2} (a+c)(b+d) + \frac{1}{2}(a-c)(b-d) $  they end up with the equations  \[
    \sum_{T \in \mathcal{T}_{h}  }^{} \left( D^2v: D^2w \right)_{T }  =
    - \sum_{E \in \mathcal{F} ^{int}}^{}  \mean{ \partial _{n_e n_{e}} v  } \jump{ \partial _{n_{e}} w } - \jump{
    \partial _{n_e n_{e}} v}  \mean{ \partial _{n_{e}} w }
    \]
\end{enumerate}


\subsubsection{HC0IP Method copied from NGSolve}%
\label{ssub:hc0ip_method_from_ngsolve}


We consider the Kirchhoff plate equation: Find $w \in H^2$, such that
$$
\int \nabla^2 w : \nabla^2 v = \int f v
$$

A conforming method requires $C^1$ continuous finite elements. But there is no good option available, and thus there is no $H^2$ conforming finite element space in NGSolve.

$$
\sum_T \nabla^2 w : \nabla^2 v
- \int_{E} \{\nabla^2 w\}_{nn} \, [\partial_n v]
- \int_{E} \{\nabla^2 v\}_{nn} \, [\partial_n w] + \alpha \int_E  [\partial_n w]  [\partial_n v]
$$

[Baker 77, Brenner Gudi Sung, 2010]

We consider its hybrid DG version, where the normal derivative is a new, facet-based variable:

$$
\sum_T \nabla^2 w : \nabla^2 v
- \int_{\partial T} (\nabla^2 w)_{nn} \, (\partial_n v - \widehat{v_n})
- \int_{\partial T} (\nabla^2 v)_{nn} \, (\partial_n w - \widehat{w_n}) + \alpha \int_E (\partial_n v - \widehat{v_n}) (\partial_n w - \widehat{w_n})
$$










