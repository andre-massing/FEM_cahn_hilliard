\newpage
\section{ Biharmonic Equation}
\label{sec:ch1}


\subsection{Strong form of the Biharmonic Equation}%
\label{sub:strong_form_of_the_biharmonic_equation}

Let $\Omega \subseteq    \mathbb{R} ^2$ be a bounded polygonal domain and $\partial \Omega $ be its corresponding boundary. Let the inhomogeneous fourth order biharmonic equation have the form,

\begin{equation}
\label{eq:bi_problem}
\begin{split}
    \Delta^2  u  + \alpha  u  & = f \quad \text{in } \Omega,   \\
    \partial _{n} u & = 0  \quad \text{on } \partial \Omega,  \\
    \partial _{n} \Delta  u & = g(x)  \quad \text{on } \partial \Omega .  \\
\end{split}
\end{equation}
Here is $\Delta ^2 = \Delta  \left( \Delta  \right) $ the biharmonic operator, also known as the bilaplacian. We will assume for the strong form that $u \in H^{4}\left( \Omega  \right) $, $\alpha $ is a non-negative constant and $f \in L_{2}\left( \Omega  \right)
$. We may consider the functions $g( x ) $ as time independent boundary conditions. Such problems as \eqref{eq:bi_problem} are often associated with the Cahn-Hilliard model
\cite{cahnhilliard1957} for phase separation. However, depending on how Cahn-Hilliard model is time discretized numerically can
\eqref{eq:bi_problem} naturally arise. I refer to \cite{brenner2012quadratic} for more information on this.

\subsection{  Weak Form Biharmonic Equation in $H^{4}\left( \Omega  \right) $}%
\label{sub:continious_weak_form_of_biharmonic_equation}


We want to introduce the full weak formulation of \eqref{eq:bi_problem}. Now, let the solution space be on the form,
\begin{equation*}
V = \left\{ v \in H^2\left( \Omega  \right) : \partial _{n} v = 0  \text{ on }
\partial \Omega  \right\}.
\end{equation*}

Let $u,v \in  V$ then the derivation of the general weak form is,
\[
\begin{split}
\left( \Delta ^2 u,v \right) _{\Omega }  &  = \left( \partial _{n} \Delta u, v \right) _{\partial \Omega } - \left( \nabla \left( \Delta  u \right) , \nabla v \right) _{\Omega }  \\
\end{split}
\]
In fact, the simplest formulation has the form,
\[
  \left( \nabla \left( \Delta u \right) , \nabla v \right) _{\Omega } =   \left( \Delta u, \partial _{n} v \right) _{\partial \Omega } - \left( \Delta u, \Delta v \right)_{\Omega },
\]
A major issue with this formulation is that we don not have boundary condition for $\Delta u$. Instead, we can expand the term in the following fashion.

\begin{equation*}
    \begin{split}
\left( \nabla \left( \Delta u \right) , \nabla v \right) _{\Omega } & = \sum_{i = 1}^{ d}  \left( \Delta  \partial _{x_{i}} u, \partial _{x_{i}}v \right) _{\Omega }  \\
&= \sum_{i = 1}^{d}  \left( \nabla \cdot \left( \nabla \partial _{x_{i}} u \right) , \partial _{x_{i}} v \right)_{\Omega }  \\
&= \sum_{i = 1}^{d}  \left( \partial_n  \partial _{x_{i}} u, \nabla  \partial _{x_{i}} v \right) _{\partial \Omega} -   \left( \nabla \partial _{x_{i}} u, \nabla \partial _{x_{i}} v \right)_{\Omega }  \\
&= \left(  \partial_n\nabla u, \nabla v \right) _{\partial \Omega } - \left( D^2 u, D^2v \right) _{\Omega } \\
&= \left( \partial _{nn} u, \partial _{n} v  \right)_{\partial \Omega }   + \left( \partial _{nt} u, \partial _{t} v \right) _{\partial \Omega } - \left( D^2u, D^2v \right) _{\Omega } .
    \end{split}
.\end{equation*}
Hence, the boundary condition of $\Delta u$ is integrated into the formulation.  It can be denoted that $D^2$ is the Hessian matrix operator such that
$$( D^2u, D^2v )_{\Omega } = \int_{\Omega }^{} D^{2}u : D^2v  dx,$$
where $D^2u:D^2v$ is the inner product and similarly for $\partial _{nn} u = n\cdot D^2 u \cdot n$. Thus, we now have a weak form identity,
\begin{equation}
\label{eq:weak_form_identity}
\left( \Delta ^2 u, v \right) _{ \Omega } = \left( D^2u, D^2v \right) _{\Omega} +   \left( \partial _{n} \Delta u, v  \right) _{\partial \Omega }  - (\partial _{nn} u, \partial _{n} v )_{\partial \Omega } - \left( \partial _{nt} u, \partial _{t}v
\right) _{\partial \Omega }
.\end{equation}

Using weak form identity \eqref{eq:weak_form_identity} and the boundary conditions stated in the strong form \eqref{eq:bi_problem} can we write

\begin{equation}
\begin{split}
\left( \Delta ^2 u, v \right) _{ \Omega } & = \left( D^2u, D^2v \right) _{\Omega} +   \underbrace{\left( \partial _{n} \Delta u, v  \right) _{\partial \Omega }}_{ = \left( g,v \right) _{\partial \Omega }}   - \underbrace{(\partial _{nn} u, \partial
    _{n} v )_{\partial \Omega }}_{ = 0}  - \underbrace{\left( \partial _{nt} u, \partial _{t}v \right) _{\partial \Omega }}_{ = 0} \\
    &= \left( D^2u, D^2v \right) _{\Omega } + \left( g,v \right) _{\partial \Omega }  \\
\end{split}
.\end{equation}

We can now formulate the full weak formulation by solving for $u \in  V$ such that
\begin{equation}
    \label{eq:bi_weak1}
a\left( u,v \right) = F(v)\quad \forall v \in
V,
\end{equation}
where
\begin{equation}
\label{eq:weak_formulation}
\begin{split}
a\left( u,v \right)_{\Omega } & =    \left( D ^2 u , D ^2 v\right)_{\Omega }  +
\alpha \left( u, v \right)_{\Omega }   , \\
F\left( v \right)_{\Omega } & = \left( f,v \right)_{\Omega } - \left(g,v \right)_{\partial \Omega }.
\end{split}
\end{equation}

 In fact, the solution is unique for $\alpha  > 0$. However, for $\alpha  = 0$ must we assume the solvability condition,
\begin{equation*}
 \int_{\Omega }^{} f dx = \int_{\partial \Omega }^{} g ds
.\end{equation*}
This condition easily arise when using the substitution $v=1$ in \eqref{eq:bi_weak1}. To handle this, can we extended the solution space \[
V^{*} = \begin{cases}
    V \quad & \alpha  > 0 \\
    \left\{ v \in V: \int_{\Omega }^{} v dx  = 0\right\} \quad & \alpha  = 0,
\end{cases}
\]
Thus, the unique solution in $v \in V^{*}$ belongs to $H^{3 }\left( \Omega  \right) $ and we get the following
elliptic regularity estimate \cite{gu2012c0},
\begin{equation}
\label{eq:bi_harmonic_ellitpic_regularity}
\left| u \right| _{H^{3 }\left( \Omega  \right) }  \le C_{\Omega } \left( \| f \|_{  L_{2}( \Omega ) }^{  } + ( 1 + \alpha  ^{\frac{1}{2}}
) \cdot \| w  \|_{ H^{4}\left( \Omega  \right)  }^{  }    \right), \quad w\in H^{4}\left( \Omega  \right).
\end{equation}
This regularity estimate may be important for further use cases in terms of error analysis.

