\section{Introduction}\label{sec:introduction}


The biharmonic equation is a fourth order partial differential equation which has gained great importance in
application such as mathematical modelling of linear elastic theory \cite{selvadurai13} and phase separation mechanics
of two phase systems \cite{cahnhilliard1957, kim16}. However, methods for solving the biharmonic equation analytically
is considered extensive and often impossible. Even for very simple plane problems on a unit square often requires advanced computations using integral transforms, variable separations, complex analysis and more \cite{selvadurai13}. We therefore tend
to lean towards approximating the solution using numerical methods for complicated problems.

There is generally two classes of numerical methods to solve the biharmonic equation. The first class is known as finite difference method (FDM) \cite{geer06,ehrlich75, hackbusch17}. Nevertheless, FDM does not handle complex domains well since it generally has strict requirements for the mesh generation. However, some methods have been introduced to solve problems on irregular domains, but is has shown to be relative extensive to implement \cite{hackbusch17, chen08, belyaev18}.

The second class is denotes as finite element method (FEM). Using this methods implies that there is theoretically no difference between solving problems on a regular or on an irregular domain, except for taking account for numerical stability and some
restrictions on mesh generation \cite{chen08}. However, a major challenge in FEM is to choose a discrete solution space on the finite elements to approximate the exact solution. We say that a method is conforming if the discrete solution space
$V_{h}$ is subspace of the exact solution space $V$, i.e. $V_{h} \subseteq  V$ \cite{shi02, brenner07math}. In general, for conforming methods requires that for a problem of order $2n$ must the discrete solution space be at least $C^{n-1}$, i.e.,
$(n-1)$ times continuously differentiable. Thus, for a biharmonic problem
will a conforming FEM method demand at least a basis that is $C^1$ globally \cite{shi02}. From this strong continuity conditions rises a lot of complexity when constructing a finite element. In fact, attempts to solve this problems has shown that it
arise 21 degrees of freedom in a triangular Argyris element \cite{nair21}.

For nonconforming methods, $V_{h} \not \subseteq  V$, the $C^{1}$ requirement is completely relaxed. This makes the methods more suitable for fourth order problems at the cost of a more extensive error analysis. In fact, designing nonconformal
elements that does converge is rather difficult \cite{shi02, nair21}.

A third approach of FEM to solve the biharmonic equation is to solve the problem applying a mixed FEM method. This method seems promising, because it only requires $C^{0}$ elements \cite{chen08, brezzi91}.
Though this works well from a numerical point of view, it has shown to have drawbacks by replacing symmetric positive definite continuous problem
by a saddle point problem, which is certainly makes the existence and uniqueness proof more challenging, as well as the design of efficient preconditioner \cite{brezzi74}.

In this report will we focus to work on a fourth approach known as the continuous interior penalty method (CIP). A major advantage is that the approach preserves the global $C^{0}$ continuity and the positive symmetric definiteness, thus makes it
attractive to solve the biharmonic equation \cite{brenner2012, brenner2012quadratic}. The goal of this report is to give a detailed account on the derivations of CIP and its a priori error analysis following the presentations in \cite{gu2012c0,
brenner2012}. We will also present a basic numerical analysis.


\section{Mathematical Background}%
\label{sub:mathematical_background}

In this section will we briefly establish some basic results of functional analysis and numerical analysis in order to construct the foundations of the FEM method. However, for a more thoroughly explanation some recommended sources are
\cite{brenner07math,manzoni2021optimal, quartdiff}. We will first define basic principles of Hilbert spaces which then will be applied to establish a theory for weak formulations and the notion of a well posed problem. Moreover, we will then continue by
establishing error estimates and constructing a numerical discretization of the weak formulation using a simple test problem.

\subsection{Hilbert Spaces}%
\label{sub:hilbert_spaces}

We will generally in section \ref{sub:mathematical_background} assume $\Omega $ to be a compact and open set in $\mathbb{R} ^{2}$. Now let the parameter $p \in \mathbb{R} $, $p\ge 1$. We then define the space $L^{p}\left( \Omega  \right) $ to be the set of all measurable functions $f: \Omega  \mapsto \mathbb{R} $ such that
$\left\lvert f \right\rvert ^{p}$ is Lebesgue measurable, i.e,

\begin{equation*}
    L^{p}\left( \Omega  \right) = \left\{ f: \Omega \mapsto \mathbb{R}  \mid \int_{\Omega }^{} \left\lvert f \right\rvert ^{p} d \Omega  < \infty  \right\}
.\end{equation*}

A useful extension, which we will use later, are the set of locally integrable functions for any compact subset $K \subseteq \text{Interior}\left( \Omega  \right) $ \cite{brenner07math}, that is,

\begin{equation*}
    L_{loc}^{1}\left( \Omega  \right)  = \left\{ f: f \in L^{1}\left( \Omega  \right)  \quad \forall K  \right\}.
\end{equation*}
Let $u \in L^{p}\left( \Omega  \right) $. We define the integral norm of order $p$ to be \[
\| u \|_{ L^{p}\left( \Omega  \right)  }^{  }  = \left( \int_{\Omega }^{} \left\lvert u \right\rvert ^{p} dx  \right) ^{\frac{1}{p}}.
\]
Since $p=2$ is frequently used in this report, we also define for convenience a compact notation $\| u \|_{ \Omega  }^{  }  = \| u \|_{ L^{2}\left( \Omega  \right)  }^{  } $ .  We say that $L^{2}\left( \Omega  \right) $ is a Hilbert space if it is equipped with a inner
product of two functions $u,v \in L^{2}\left( \Omega  \right) $ such that
\[
\left( u,v \right) _{\Omega } = \left( u,v \right) _{L^2\left( \Omega  \right) } = \int_{\Omega }^{} u  v dx.
\]

To generalize, we denote the notation $\mathcal{V} $ for a arbitrary Hilbert space. Furthermore, we define the dual space the be the space of linear and bounded functionals $F: \mathcal{V}  \mapsto \mathbb{R} $\cite{quartdiff}, i.e., \[
\mathcal{V} ^{*} =
\left.
\begin{cases}
F: \mathcal{V}  \mapsto \mathbb{R} \text{ such that }\forall v,w \in \mathcal{V}, \forall a,b \in \mathbb{R} \text{ and } C> 0 \text{ is }   \\
  F\left( \lambda v + \mu w  \right) = \lambda F(v) + \mu F(w) \text{ and } \left\lvert F\left( v \right)  \right\rvert \le C \| v \|_{ \mathcal{V}  }^{  }
\end{cases}
  \right\}
\]
and we equip it with the functional norm,  \[
    \| F \|_{ \mathcal{V} ^{*} }^{  } = \sup_{v \in \mathcal{V}   } \frac{\left\lvert F\left( v \right)  \right\rvert }{\| v \|_{ \mathcal{V}  }^{  } }.
\]

We will now establish a notion of the weak derivative, but first are we going to characterize some useful definitions of continuity. The space $C^{k}\left( \Omega  \right) $ for $k\ge 0$ denotes the set of functions whose derivatives, up to order of
$k$ , is continuous in $\Omega $. Note that we often use the shorthand notation $ C^{0} = C\left( \Omega  \right)  = C^{0}\left( \Omega  \right) $.
From this, let $C^{\infty}\left( \Omega  \right) $ be the set of infinitely differentiable functions in $\Omega $. Furthermore, we then denote the space $C^{\infty}_{0}\left( \Omega  \right)$ as the space of all functions, $u \in C^{\infty}\left( \Omega
\right) $, vanishing outside of any compact subset of $\Omega $. Let $u,v \in  C^{1}\left( \Omega  \right) $ and the define boundary $\Gamma  = \partial \Omega $ with a corresponding outer normal vector $n$. It is well known that this partial
integration formula holds \cite{manzoni2021optimal},

\[
\int_{\Omega }^{} \nabla u \cdot v dx = \int_{\Gamma }^{} u\cdot v n ds - \int_{\Omega }^{} u \cdot \nabla v dx.
\]
We now use this notation for derivatives
\footnote{In literature is often $D^{\alpha } f$ commonly used, but later in the report is this notation reserved for the Hessian operator. Therefore, we then the notation $\partial ^{\alpha } f$ in this report.} so
\begin{equation}
\label{eq:mixed_derivative}
\partial ^{\alpha  } f = \frac{\partial ^{\left\lvert \alpha  \right\rvert } f}{ \partial ^{\alpha _{1} } x_{1} \partial ^{\alpha _{2}} x_{2}  }, \quad \text{where } \alpha=\left( \alpha _{1}, \alpha _{2} \right) \text{ and } f \in C^{\left\lvert \alpha  \right\rvert }
\left( \Omega  \right)
.\end{equation}
Finally, let $u \in  L^{1}_{loc}\left( \Omega  \right) $. We call the function $w \in L_{loc}^{1}\left( \Omega  \right) $ the $\alpha $-th weak derivative of $u$  if \[
\int_{\Omega }^{} w \varphi  dx = \left( -1 \right) ^{\left\lvert \alpha  \right\rvert } \int_{\Omega }^{} u \cdot \partial ^{\alpha } \varphi dx, \quad \forall \varphi \in  C_{0}^{\infty}\left( \Omega  \right).
\]

We are now able to construct the Sobolev space \cite{manzoni2021optimal}, \[
H^{m}\left( \Omega  \right) = \left\{ u \in L^{2}\left( \Omega  \right)  \mid  \partial ^{\alpha } u \in L^{2}\left( \Omega  \right)  \forall \alpha : \left\lvert \alpha  \right\rvert  \le m \right\} \text{ for } m>1
\]
Equipped with the inner product is $H^{m}\left( \Omega  \right) $  denoted as a Hilbert space, that is, for $u,v \in H^{m}\left( \Omega  \right) $, \[
    \left( u,v \right) _{H^{m}\left( \Omega   \right) } = \sum_{\left\lvert \alpha  \right\rvert  \le  m}^{}  \int_{\Omega }^{} \partial ^{\alpha } u \partial ^{\alpha } v dx.
\]
Similarly, the integral norm is denoted as, \[
\| u \|_{ H^{m}\left( \Omega  \right)  }^{  }  = \left( \| u \|_{ L^{2}\left( \Omega  \right)    } + \sum_{k = 1}^{m}  \left\lvert u \right\rvert ^{2} _{  H^{k}\left( \Omega  \right) }\right) ^{\frac{1}{2}},
\]
where the seminorm is defined such that, \[
\left\lvert u \right\rvert _{H^{k}\left( \Omega  \right) } = \left( \sum_{\left\lvert \alpha  \right\rvert  = k}^{} \| \partial ^{\alpha }u \|_{ \Omega  }^{ 2 }  \right).
\]
For convenience, we also entitle the notation,
\[
H^{m}_{0} \left( \Omega  \right) = \left\{ \text{completion of }C_{0}^{\infty}\left( \Omega  \right) \text{ w.r.t. } \| \cdot  \|_{H^{m}\left( \Omega  \right)   }^{  }  \right\}.
\]

\subsection{Weak Formulation}%
\label{sub:weak_formulation}

By applying the theory of weak derivative it it possible describe the notion of a so-called weak formulation of a system. Let us consider a simple test problem, that is, the strong form of the Poisson's equation with homogeneous Dirichlet boundary
conditions,
\begin{equation}
\label{eq:poisson_strong}
\begin{split}
-\Delta u & = f \quad \text{in } \Omega \\
u & =0 \quad \text{on } \partial \Omega
\end{split}
.\end{equation}
We want to obtain the weak formulation of the problem \eqref{eq:poisson_strong}. Let the trial function be $u \in H_{0}^{1}\left( \Omega  \right) $, then by integrating with a test function we get, \[
\int_{\Omega }^{} - \Delta u \cdot v dx = \int_{\Omega }^{} f \cdot v dx \quad \forall v \in H^{1}_{0}\left( \Omega  \right).
\]
By using the partial integration formula and take the advantage of homogeneous Dirichlet boundary conditions the equation can the formulation be simplified to \[
a\left( u,v \right) = \int_{\Omega }^{}  \nabla u\cdot \nabla v dx , \quad l\left( v \right)  = \int_{\Omega }^{}  f v dx.
\]
Thus, we say the final weak formulation is the problem where we want to find a solution $u \in H^{1}_{0}\left( \Omega  \right) $  such that it fulfills the criteria
\begin{equation}
\label{eq:poissons_weak_formulation}
a\left( u,v \right) = l\left( v \right) \quad \forall v \in H^{1}_{0}\left( \Omega  \right).
\end{equation}

We now want to establish that problems on this form is well-posed.
Lax-Milgram Theorem says that if we assume a Hilbert space $\mathcal{V} $ and a bilinear symmetric problem, where we want to find a $u \in \mathcal{V} $,  such that

\begin{equation*}
    a\left( u,v \right)  = \left( F, v \right) _{\mathcal{V} ^{*}, \mathcal{V} }, \quad \forall v \in  \mathcal{V} \text{ and } F \in \mathcal{V} ^{*}
.\end{equation*}
 Assume the bilinear form, $a\left( \cdot ,\cdot  \right) $, is continuous and coercive, i.e., the two conditions below.
\begin{enumerate}[label=\arabic*)]
    \item There exists a constant $M>0$ so that,  \[
    \left\lvert a\left( v,w \right)  \right\rvert \le M \| v \|_{ \mathcal{V}  }^{  } \| w \|_{ \mathcal{V}  }^{  }  \quad \forall v,w \in \mathcal{V}.
    \]
\item There exists a $\alpha  > 0$  so, \[
a\left( v,v \right)  \ge  \alpha \| v \|_{ \mathcal{V}  }^{ 2 }.
\]
\end{enumerate}
If this is true, then there exists a unique solution, $u \in \mathcal{V} $, which satisfies the stability estimate \cite{manzoni2021optimal}, \[
\| w \|_{ \mathcal{V}   }^{  } \le  \| F \|_{ \mathcal{V} ^{*} }^{  }.
\]
Problems that fulfills this criteria is said to be well posed.

Moreover, we can clearly see that this theorem is applicable to the weak formulation \eqref{eq:poissons_weak_formulation}.
\[
\begin{split}
    \left\lvert a\left( v,w \right)  \right\rvert & \le \| \nabla v \|_{ \Omega  }^{  } \| \nabla w \|_{ \Omega  }^{  } \le \| v \|_{ \mathcal{V}  }^{  } \| w \|_{\mathcal{V}   }^{  }   \\
 a\left( v,v \right)    & = \frac{1}{2} \| \nabla v  \|_{ \Omega  }^{  2} + \frac{1}{2} \| \nabla v  \|_{ \Omega  }^{  2} \ge \beta \left( \| v \|_{\Omega   }^{ 2 }  + \| \nabla u \|_{\Omega   }^{ 2 } \right) \ge \beta \| v \|_{\mathcal{V}   }^{  }
\end{split}
\]
 Here is $\beta = \min_{} \left\{ \frac{1}{2}, \frac{1}{2C^{2}} \right\} $. In the last inequality was the Poincare inequality applied, i.e., there exists a $C>0$  such that $\| v \|_{ \Omega  }^{  } \le C  \| \nabla v \|_{\Omega   }^{  } $. Hence, the weak formulation \eqref{eq:poissons_weak_formulation} is well posed.

\subsection{Ceas' Lemma}%
\label{sub:ceas_lemma}

Since we have established a theory of a well-posed weak formulation we can now transition to setup a theory for a approximate solution. Assume that we have a problem that satisfies the Lax-Milgram theorem and let $\mathcal{V} _{h} \subseteq  \mathcal{V} $  be
some finite dimensional subspace of $\mathcal{V} $ such that $dim\left( \mathcal{V} _{h} \right) =N_{h}$ , where $h$  is a discretization parameter. We transition the problem to find a solution, $u_{h} \in  \mathcal{V}_{h}$, such that,
$$a\left( u_{h},v_{h} \right)  =
l\left( v_{h} \right) \quad  \forall v_{h} \in \mathcal{V} _{h} .$$
Since the method is conform, i.e., $\mathcal{V} _{h} \subseteq  \mathcal{V} $, does it exists a exact solution, $u \in  \mathcal{V} _{h}$, so, \[
a \left( u, v_{h} \right)  = l\left( v_{h} \right)  \quad  \forall v_{h} \in  \mathcal{V} _{h}.
\]
Furthermore, the problem is said to be strongly consistent since it fulfills the Galerkin orthogonality property, that is,
$$ a\left( u -u_{h} , v_{h} \right)  =0.$$

Now we have all the cornerstones for computing a error estimate. Let $v_{h} \in  \mathcal{V} _{h}$ so
\begin{equation}
\label{eq:cealemma_proof}
    \begin{split}
\alpha \| u -u_{h} \|_{ \mathcal{V}  }^{ 2 } & \le  a\left( u - u_{h}, u - u_{h}  \right)    \\
&= a\left( u - u_{h}, u -v_{h} \right) - a\left( u -u_{h}, v_{h} - u_{h} \right)  \\
 &  \le  M \| u - u_{h} \|_{ \mathcal{V}  }^{  }  \| u - v_{h} \|_{ \mathcal{V}  }^{  }
    \end{split}
.\end{equation}
Hence, we now have the Ceas' lemma \cite{quartdiff}, \[
\| u - u_{h} \|_{ \mathcal{V}  }^{  }  \le  \frac{M}{\alpha } \inf_{v_{h} \in \mathcal{V} } \|  v_{h} - u \|_{  }^{  }.
\]
A useful property is that for a conformal numerical method to converge can we now simply require \[
\lim_{h \to 0}  \inf_{v_{h} \in  V_{h}}  \| v - v_{h} \|_{ \mathcal{V }  }^{  } = 0 \quad  \forall v \in \mathcal{V}.
\]
In that case will $\| u - u_{h} \|_{ \mathcal{V}  }^{  }  \to  0$, $h \to  0$. Hence, if this requirement is fulfilled, the numerical methods will converge to a unique solution.



\subsection{Finite Element Method}%
\label{sub:finite_element_method}
The idea of a conformal FEM method is to approximate a solution space, $\mathcal{V}_h \subseteq \mathcal{V}  $. However, the building blocks of the discretization relies on constructing a triangulation $\mathcal{T } _h$ with non-overlapping triangles, $T \in
\mathcal{T}_{h} $, so that, \[
\Omega _{h} = Interior(\bigcup_{T \in  \mathcal{T} _{h}}^{} T) \implies \lim_{h\to 0} measure(\Omega - \Omega _{h}) = 0
\]
For convenience, we will generally use the notation $\Omega  = \Omega  _{h}$ \footnote{Some measure theorists  might be upset for this informality.}. We also denote the parameter $h$ as the diameter of the triangle $T$, i.e., $h_{T} = diam(T)$. Anyhow, this is later more precisely in the subsection
\ref{sub:computational_domain}, but as a basic introduction to FEM is this sufficient.
Furthermore, let the space of finite elements suitable for our test problem in subsection \ref{sub:weak_formulation} to be, by using the definition from \cite{quartdiff},
\[
\mathcal{V} _{h} = \left\{ v_{h} \in C^{0}\left( \Omega  \right)  \mid  v_{h} \in \mathcal{P} _{k}\left( T \right)  \quad \forall T \in \mathcal{T} _{h}  \right\},
\]
where $\mathcal{P } _{k}\left( T \right) $ is the space of polynomials with degree $k$ for each $T$, that is,
\[
\mathcal{P} _{k}\left( T \right)  = \left\{ p\left( x_{1}, x_{2} \right) = \sum_{}^{i + j}  a_{ij} x^{i}_{1} x^{j}_{2}  \mid  a_{ij} \in \mathbb{R}  \text{ and } i,j \ge  0   \right\}.
\]

For the test problem can we observe that it is convenient to deal with a Lagrangian functions $\varphi _{j} \in V_{h}$. Thus, for every node $N_{j}$, \[
\varphi _{j}\left( N_{i} \right)  = \delta _{ij} = \begin{cases}
    0, \quad & i \neq j \\
    1,\quad & i=j
\end{cases}
\text{ where } i,j = 1,\ldots, N_{h}.
\]
Here is $N_{h}$  the total number of nodes which is chosen with caution. For the test problem, where $\Omega \subseteq \mathbb{R} ^{2} $, then for $k=1$ the nodes is typically defined on the vertices and for $ k=2 $ is the nodes defined on the
vertices and the center of the facets \cite{quartdiff}.

Finally, by using the set of basis functions, $\left\{ \varphi _{j} \right\} $, can we discretize a solution, $u_{j} = u_{h}\left( N_{j} \right) $, so that $u_{h} = \sum_{j}^{N_{h}} u_{j} \varphi _{j}\left( x_{1},x_{2} \right)  $ fulfills the
criteria, \[
\sum_{j = 1}^{N_{h}} u_{j} a\left( \varphi _{j}, \varphi _{i} \right)  = l\left( \varphi _{i} \right)
\]
Hence, by letting $\mathbf{u} = \left[ u_{j} \right] $ , $\mathbf{f} = \left[ \left( f, \varphi _{i}  \right) _{\Omega } \right] $  and $A = \left[ a\left( \varphi _{j}, \varphi _{i} \right)  \right] $ can we construct a linear system, \[
A \mathbf{u} =\mathbf{f}.
\]
Ultimately, we have the final linear system which can be solved for each unknown $u_{j}$.
















