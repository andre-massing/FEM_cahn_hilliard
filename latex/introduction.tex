\section{Introduction}\label{sec:introduction}


The biharmonic equation is a fourth order partial differential equation which has gained great importance in
application such as mathematical modelling of linear elastic theory \cite{selvadurai13} and phase separation mechanics
of two phase systems \cite{cahnhilliard1957, kim16}. However, methods for solving the biharmonic equation analytically
is considered extensive and often impossible. Even on very simple plane problems on a unit square often requires advanced computations using integral transforms, variable separations, complex analysis and more \cite{selvadurai13}. We therefore tend
to lean towards approximating the solution using numerical methods for complicated problems.

There is generally two classes of numerical methods to solve the biharmonic equation. The first class is known as finite difference method (FDM) \cite{ehrlich75, hackbusch17}. Nevertheless, FDM does not handle complex domains well since it generally has strict requirements for the mesh generation. However, some methods have been introduced to solve problems on irregular domains, but is has shown to be relative extensive to implement \cite{hackbusch17, chen08, belyaev18}.

The second class is denotes as finite element method (FEM). Using this methods implies that there is theoretically no difference on solving problems on a regular or irregular domains, except for taking account for numerical stability and some
restrictions on mesh generation \cite{chen08}. However, a major challenge in FEM is to choose a discrete solution space on the finite elements to approximate the exact solution. We say that a method is conforming if the discrete solution space
$V_{h}$ is subspace of the exact solution space $V$, i.e. $V_{h} \subseteq  V$ \cite{shi02, brenner07math}. In general, for conforming methods requires that for a problem of order $2n$ must the discrete solution space be at least of order $n-1$. Thus, for a biharmonic problem
will a conforming FEM method demand at least a basis that is $C^1$ globally \cite{shi02}. From this strong continuity conditions rises a lot of complexity when constructing a finite element. In fact, attempts to solve this problems has shown that it
arise 21 degrees of freedom in a triangular element \cite{nair21}.

For nonconforming methods, $V_{h} \not \subseteq  V_{h}$ is the $C^{1}$ requirement completely relaxed. This makes the methods more suitable for forth order problems with the cost of more extensive error analysis. In fact, designing nonconformal
elements that does converge is rather difficult \cite{shi02, nair21}.

A third approach of FEM to solve the biharmonic equation is to solve the problem doing a Mixed FEM method. This method seems promising, because it only require $C^{0}$ elements \cite{chen08, brezzi91}.
Though this work well from a numerical point of view, but it has shown to have drawbacks by replacing symmetric positive definite continuous problem
by a saddle point problem, which is certainly makes the existence and uniqueness proof more challenging \cite{brezzi74}.

In this report will we focus to work on a fourth approach based on a FEM method called the continuous interior penalty method (CP). A major advantage is that the approach preserves the global $C^{0}$ continuity and the positive symmetric definiteness, thus makes it attractive to solve the biharmonic equation \cite{brenner2012, brenner2012quadratic}. In this report will we focus on presenting the derivation of CP and carry out a basic error analysis. We will also present a numerical analysis.

